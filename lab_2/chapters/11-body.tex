\justifying
\textbf{Цель работы:}
Научиться решать системы линейный алгебраических уравнений итерационными методами Якоби, Зейделя и SOR.

\textbf{Задание:}
Решить систему линейных алгебраических уравнений итерационными методами Якоби, Зейделя и SOR методом
\begin{equation}\label{task}
    Ax = f,\quad A = 
    \begin{pmatrix}
    10 & 3 & 0 \\
    3 & 15 & 1 \\
    0 & 1 & 7
    \end{pmatrix}, \quad
    f = 
    \begin{pmatrix}
    2 \\
    12 \\
    5
    \end{pmatrix}.
\end{equation}
с относительной погрешностью, удовлетворяющей
\begin{equation}
    \frac{\|\bar{x} - x^{(n)} \|}{\| \bar{x} \|} \leq \varepsilon,
\end{equation}
где $\bar{x}$ есть точное решение, $\bar{x}^{(n)}$ -- итерация с номером $n$, а $\varepsilon$ задано (например, $\varepsilon = 10^{-4}$). Сравнить количество итераций для всех трех методов.

\textbf{Ход работы:}

Представим исходную матрицу системы в виде суммы трех матриц:
\begin{equation}
    A = A_1 + D + A_2,
\end{equation}
где $D$ -- матрица, состоящая только из диагонали, $A_1$ -- нижняя треугольная матрица с нулями на диагонали и выше, $A_2$ -- верхняя треугольная матрица с нулями на диагонали и ниже.

С учетом такого разложения матрицы $A$, метод Якоби записывается в следующей итерационной форме:
\begin{equation}
    x^{(n+1)} = -D^{-1}A_1x^{(n)} -D^{-1}A_2x^{(n)} + D^{-1}f
\end{equation}

Метод Зейделя записывается, как:
\begin{equation}
    x^{(n+1)} = -(D + A_1)^{-1}A_2x^{(n)} + (D + A_1)^{-1}f
\end{equation}

Метод SOR получается из метода Зейделя, если введен дополнительный параметр $\omega$:
\begin{equation}
    x^{(n+1)} = (I + \omega D^{-1}A_1)^{-1}[(1 - \omega)I - \omega D^{-1}A_2]x^{(n)} + (I + \omega D^{-1}A_1)^{-1}\omega D^{-1}f
\end{equation}

Если реализовать указанные алгоритмы на Python с $\varepsilon = 10^{-4}$ и взять начальное приближение
\begin{equation}
    x_0 = \begin{pmatrix}
    1 \\
    1 \\
    1
    \end{pmatrix},
\end{equation}
то получатся следующие результаты:
\begin{enumerate}
    \item Метод Якоби сходится за 7 итераций и дает решение
    \begin{equation}
        x = \begin{pmatrix}
        -0.02970633 \\
        0.76553095 \\
        0.60490175
        \end{pmatrix}
    \end{equation}
    \item Метод Зейделя сходится за 5 итераций и дает решение
    \begin{equation}
        x = \begin{pmatrix}
        -0.02968127 \\
        0.76560425 \\
        0.60492278
        \end{pmatrix}
    \end{equation}
    \item Метод SOR с $\omega = 1.01$ сходится за 4 итерации и дает решение
    \begin{equation}
        x = \begin{pmatrix}
        -0.02967632 \\
        0.76558363 \\
        0.60497565
        \end{pmatrix}
    \end{equation}
\end{enumerate}

\textbf{Выводы:}

В настоящей лабораторной работе была решена система линейных алгебраических уравнений~(\ref{task}) с помощью итерационных методов Якоби, Зейделя и SOR. Получилось, что медленнее всего сходится метод Якоби, потом -- метод Зейделя. Быстрее всего сошелся метод SOR с параметром $\omega = 1.01$. Следует отметить, что при других значениях параметра $\omega$ итерационный процесс может сходиться значительно медленнее.