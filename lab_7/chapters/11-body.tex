\justifying
\textbf{Цель работы:}
Научиться вычислять двойные интегралы методом Монте-Карло.

\textbf{Задание:}
Вычислить двойной интеграл
\begin{equation}\label{init_int}
    I = \int_{x ^ 2 + y ^ 2 \leq 1} (A x ^ 2 + B y ^ 2 + C) dx dy, \quad A > 0, \quad B > 0, \quad C > 0,
\end{equation}
методом Монте-Карло двумя способами. Первый способ основан на применении теоремы о среднем, второй способ использует представление двойного интеграла как объема вертикального цилиндра, ограниченного сверху поверхностью $z = f(x, y) = A x ^ 2 + B y ^ 2 + C$. Сравнить результаты вычислений.

\textbf{Ход работы:}

Пусть
\begin{equation}
    A x ^ 2 + B y ^ 2 + C = f(x, y),
\end{equation}
тогда 
\begin{equation}\label{small_int}
    I = \int_{x ^ 2 + y ^ 2 \leq 1} f(x, y) dx dy.
\end{equation}

Первый способ основан на теореме о среднем:
\begin{equation}
    \int_{x ^ 2 + y ^ 2 \leq 1} f(x, y) dx dy = f(\Bar{x}, \Bar{y}) S,
\end{equation}
где $(\Bar{x}, \Bar{y}) \in x ^ 2 + y ^ 2 \leq 1$, а $S$ -- площадь области $x ^ 2 + y ^ 2 \leq 1$. В качестве приближения возьмем $f(\Bar{x}, \Bar{y})$ за среднее арифметическое значений функций $f(x, y)$ в $n$ случайных точках, попавших в область $x ^ 2 + y ^ 2 \leq 1$. Площадь приблизим следующим выражением:
\begin{equation}
    S \approx (b - a)(d - c) \frac{n}{N},
\end{equation}
где $a, b, c, d$ -- границы области $x ^ 2 + y ^ 2 \leq 1$, равные $1$ и $-1$, $n$ -- количество попавших в нужную область точек, $N$ -- общее количество точек, распределенных на плоскости.

Таким образом интеграл~(\ref{small_int}) переходит в~(\ref{first_final}).
\begin{equation}\label{first_final}
    \int_{x ^ 2 + y ^ 2 \leq 1} f(x, y) dx dy \approx \frac{(b - a)(d - c)}{N} \sum_{k = 1} ^ {n} f(x_k, y_k).
\end{equation}

Второй метод основан на представление двойного интеграла как объема вертикального цилиндра, ограниченного сверху поверхностью $z = f(x, y)$. В таком случае формулы будут следующими:
\begin{align}
    \int_{x ^ 2 + y ^ 2 \leq 1} f(x, y) dx dy &\approx (b - a)(d - c) M \frac{n}{N}, \\
    M &= \max_{(x, y) \in D_{xy}} f(x, y),
\end{align}
где $a, b, c, d$ -- границы области $x ^ 2 + y ^ 2 \leq 1$, равные $1$ и $-1$, $n$ -- количество попавших в нужную область точек, $N$ -- общее количество точек, распределенных на плоскости.

При следующих параметрах $A = 6, B = 8, C = 7, N = 10 ^ 6, n = 10 ^ 3$ первый метод дает результат $32.888528$, второй -- $32.894736$, а точное решение -- $32.986723$.

\textbf{Выводы:}

В настоящей работе был вычислен интеграл~(\ref{init_int}) двумя методами Монте-Карло. Полученные результаты указывают на то, что методы Монте-Карло достаточно хорошо аппроксимируют точное значение кратных интегралов. В рамках настоящей лабораторной работы второй метод получился ближе к точному решению на~$0.01$.